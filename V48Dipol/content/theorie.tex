\section{Zielsetzung}
In diesem Versuch geht es darum, die Relaxationszeit in Abgängigkeit der Temperatur und die Aktivierungsenergie für Diffusion der Leerstellen
von Kaliumbromit, welches mit Strontium dotier ist, zu finden. 

\section{Theorie}


\begin{equation}
\tau \left( \text{T} \right) = \tau_0  \text{exp} \left(\frac{\text{W}}{\text{k}_{\text{b}}\text{T}}\right) 
\label{eqn:relaxo}
\end{equation}

\begin{equation}
\text{j} \left( \text{T} \right) = \text{y} \left( \text{T}_{\text{p}}\right) \text{p} \frac{\text{dN}}{\text{dt}}
\label{eqn:dichte}
\end{equation}

\begin{equation}
\text{N}\left( \text{T} \right) = \text{N}_{\text{p}} \text{exp} \left( - \frac{1}{\text{b}} \int^{\text{T}}_{\text{T}_0} \frac{\text{dT'}}{\text{dt}}\right)
\label{eqn:orientierung}
\end{equation}

\begin{equation}
\text{j} \left( \text{T} \right) = \frac{\text{p}^2 \text{E}}{3\text{k}_{\text{b}}\text{T}_{\text{p}}} \frac{\text{N}_{\text{p}}}{\tau} \text{exp} \left( -\frac{1}{\text{b}} \int^{\text{T}}_{\text{T}_0} \text{dT'} \text{exp} \left( - \frac{\text{W}}{\text{k}_{\text{B}}\text{T'}} \right) \text{dt} \right) \text{exp} \left(-\frac{\text{W}}{\text{k}_{\text{b}}\text{T}} \right)
\label{killme}
\end{equation}

\begin{equation}
\end{equation}

\begin{equation}
\end{equation}

\begin{equation}
\end{equation}