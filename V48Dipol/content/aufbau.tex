\section{Aufbau und Durchführung}
\szbection{Aufbau}
\begin{figure}
    \centering
    \includegraphics{width=0.5/textwidth}{bilder/aufbau.png}
    \caption{"Schematischer Aufbau einer Anlage zur Messung der Dipolrelaxation.
            \cite{skript}}
    \label{fig:aufbau}
\end{figure}

Zentrales Element zur Messung ist die Probe selbst, die in den Rezipent gelegt wird. Diese gilt es im Verlauf entsprechend
zu erwärmen und zu kühlen. Als Probe lassen sich Kaliumbromid oder CsJ verwenden. Diese sind mit Strontium
otiert. Die Aufwärmung erfolgt durch ein Heiznetzgerät, welches die Termperatur aber nur anheben kann.
Hilfreich ist es diesen Prozess durch Wärmeleitpaste zu unterstützen um
Verlust durch nicht gut verbundene Kontaktflächen zu vermeiden.
Nötig ist es außerdem ein hinreichend gutes Vakuum von \SI{10**-2}{\milli\bar} durch eine angeschlossene Vakuumpumpe zu erzeugen.
Eventuelle Druckmessungen  lassen sich durch ein Messgerät zwischen der Pumpe und dem Vakuum tätigen.
Oberhalb der Probe und untherlab des Vakuumbeälters gilt es nun eine Spannung anzulegen um so ein 
elektrisches Feld (um die Probe herum) zu erzeugen. Um die Temperatur zu messen bietet sich ein, am Aufbau angebrachtes, Thermometer an.
Kontrer zur Heißvorrichtung der Probe, lässt sie sich auch durch einen Kühlfinger abkühlen.
Dieser Kühlfinger wird bei etwaigen Kühlungen in den Dewar, 
der mit flüssigem Stickstoff fefüllt ist, 
zu tauchen und die Temperatur in der Probe verliert, unter anderem durch die Wärmeleitpaste, so schnell an Wert.
Ein empfindliches Picoampermeter steht außerdem bereit im richitgen Moment auch kleine
Stromflüsse zu messen.

\begin{figure}
    \centering
    \includegraphics{width=0.5/textwidth}{bilder/aufbau2.png}
    \caption{"Schematischer Aufbau einer Anlage zur Messung der Dipolrelaxation mit 
            besonderem Fokus auf den Rezipten.
            \cite{skript}}
    \label{fig:aufbau2}
\end{figure}


Im Rezipenten selbst liegt auf der Probe noch eine Metallplatte um das elektrische Feld anlegen zukönnen.
Dieses Feld wir mit einer Spannung von bis zu $\SI{900}{\volt}$ aufrecht gehalten.
\\
\newline
\subsection{Durchführung}
Zunächst gilt es ein Vakuum im Rezipient zu erzeugen. Dafür wird die Drehschieber-Vakuumpumpe
aktiviert die in der Lage ist einen Unterdruck von bis zu $\SI{10**-2}{\milli\bar}$ aufrecht zu halten.
Dazu muss die Temperatur der Probe maßgeblich erhöt werden, auf bis zu $\SI{295}{\kelvin}$ für 
CsJ und $\SI{320}{\kelvin}$ für KBr. Kontrolliert wird der Wert mit einem Thermometer.
Nun wird eine Spannung von $\SI{950}{\volt}$ an die Metallplatte, bzw. den Kondensator, angelegt. Es entsteht also ein elektrisches 
Feld durch die Probe hindurch. Der Kondensator braucht einige Zeit um vollständig geladen zu sein,
so soll also etwa $\SI{900}{\second}$ gewaret werden um eine hinreichende Ladung zu garantieren.
Außerdem ist die so gewählte Ladezeit groß gegenüber der Relaxationszeit $\tau$.
