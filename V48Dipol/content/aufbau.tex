\section{Aufbau und Durchführung}
\subsection{Aufbau}
\begin{figure}
    \centering
    \includegraphics[width=0.8\textwidth]{bilder/Aufbau.png}
    \caption{Schematischer Aufbau einer Anlage zur Messung der Dipolrelaxation nach
            \cite{skript}.}
    \label{fig:aufbau}
\end{figure}

Zentrales Element zur Messung ist die Probe selbst, die in den Rezipenten gelegt wird. Diese gilt es im Verlauf entsprechend
zu erwärmen und zu kühlen. Als Probe lässt sich Kaliumbromid, der mit Strontium
dotiert ist, verwenden. Die Aufwärmung erfolgt durch ein Heiznetzgerät, welches die Heizrate durch Veränderung der angelegten Spannung ändern kann.
Hilfreich ist es diesen Prozess durch Wärmeleitpaste zu unterstützen um
Verlust durch nicht gut verbundene Kontaktflächen zu vermeiden.
Nötig ist es außerdem ein hinreichend gutes Vakuum in einem Rezipienten von $\SI{0.1}{\milli\bar}$ 
durch eine angeschlossene Vakuumpumpe zu erzeugen.
Eventuelle Druckmessungen  lassen sich durch ein Messgerät zwischen der Pumpe und dem Vakuum tätigen.
Oberhalb und unterhalb der Probe lässt es sich durch einen Plattenkondensator eine Spannung anlegen um so ein 
elektrisches Feld (um die Probe herum) zu erzeugen. Um die Temperatur zu messen bietet sich ein, am Aufbau angebrachtes, Thermometer an.
Konträr zur Heizvorrichtung, lässt sich die Probe durch einen Kühlfinger, der in ein mit flüssigem Stickstoff gefülltes Dewargefäß getaucht wird, 
abkühlen.
Ein empfindliches Picoampermeter, angeschlossen an den Kondensatorplatten, steht außerdem bereit im richitgen Moment auch kleine
Stromflüsse zu messen.

\begin{figure}
    \centering
    \includegraphics[width=0.8\textwidth]{bilder/aufbau2.png}
    \caption{Schematischer Aufbau einer Anlage zur Messung der Dipolrelaxation mit 
            besonderem Fokus auf den Rezipienten nach 
            \cite{skript}.}
    \label{fig:aufbau2}
\end{figure}


Im Rezipienten selbst liegt auf der Probe noch eine Metallplatte um das elektrische Feld anlegen zukönnen.
Dieses Feld wir mit einer Spannung von bis zu $\SI{950}{\volt}$ aufrecht gehalten.
\\
\newline
\subsection{Durchführung}
%Zunächst gilt es ein Vakuum im Rezipienten zu erzeugen. Dafür wird die Drehschieber-Vakuumpumpe
%aktiviert die in der Lage ist einen Unterdruck von bis zu $\SI{10e-2}{\milli\bar}$ aufrecht zu halten.
%Dazu muss die Temperatur der Probe maßgeblich erhöht werden, auf bis zu $\SI{320}{\kelvin}$ für Kaliumbromit. Kontrolliert wird der Wert mit einem Thermometer.
Zu Beginn wird eine Spannung von $\SI{950}{\volt}$ an die Metallplatte, bzw. den Kondensator, angelegt. Es entsteht also ein elektrisches 
Feld durch die Probe hindurch. Die Dipole brauchen einige Zeit um vollständig ausgerichtet zu sein,
so soll also etwa $\SI{900}{\second}$ gewartet werden.
Außerdem ist die so gewählte Ladezeit groß gegenüber der Relaxationszeit $\tau$, damit möglichst viele Dipole ausgerichtet sind.
Sobald die Zeit abgelaufen ist, es bietet sich an diese mit einer Stoppuhr zu messen,
gilt es die Probe mit flüssigen Stickstoff auf etwa $\SI{210}{\kelvin}$ zu kühlen.
Bevor Messungen stattfinden dürfen, muss die elektrische Spannung ausgeschaltet werden.
Da der Kondensator ein Teil seiner Ladung, auch nach herunterregeln des Netzgerätes, behalten wird,
muss dieser Kurzgeschlossen werden und es wird gewartet, bis er die von ihm erzeugte Spannung verloren hat.
Nach einigen Minuten darf das Picoampermeter angeschlossen werden. Jedoch muss,
wenn dies noch nicht der Fall ist, darauf gewartet werden bis der Stromwert konstant ist.
Mit der Heizstromversorgung soll die Probe nun wieder erwärmt werden, wobei die 
Heizrate, die konstant sein soll, gut beobachtet werden muss. Aus der thermodynamischen Natur 
folgt, dass es bei ansteigender Tempatur immer mehr Arbeit braucht um etwaige Materie bei 
konstanter Umgebungstemperatur zu erwärmen. Also sei der Heizstrom variabel am Gerät
einstellbar und muss im Verlauf angepasst, bzw. erhöht werden.
\\
\newline
Die wichtigen Messdaten für den Versuch die es zu Messen gilt, ist der am Picoampermeter 
abgelesene Strom beim Aufheizen und die dabei auftretende Temperatur der Probe.
Es werden zwei Messreihen gestartet, wobei die Heizrate von ursprünglich $\SI{1}{\degree\per\minute}$
auf $\SI{2}{\degree\per\minute}$ variiert wird.