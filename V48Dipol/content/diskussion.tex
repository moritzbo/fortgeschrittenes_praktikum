\section{Diskussion}


Die Temperatur weicht besonders am Anfang von der vorgenommenen Heizrate ab. Das liegt daran, dass die Probe noch
gekühlt wird und die Heizdrähte noch nicht die erhöhte Temperatur angenommen haben. Bei den späteren Messungen passt sich der Verlauf immer besser an
und wird hinreichend gut durch die Veränderung an der Heizspannung korriegert. In der Abbildung \ref{fig:a1} sieht man einen guten linearen Verlauf.
Das Dewargefäß wurde während des Aufheizen nicht von dem Kühlfinger entfernt, weswegen gerade zu Beginn mit erhöhter Heizspannung erwärmt wurde.
\\
\newline
Die gemittelten, nach den zwei Methoden gefundenen, Aktivierungsenergien $\overline{W_{b}}$, mit Heizrate $b$,  liegen sehr nahe bei beieinander und weisen folgende Abweichung $W_{i,\%}$ zu dem Theoriewert $W_{\text{theo}}$, nach \cite{https://doi.org/10.1002/pssb.2220610223}, auf.
\begin{align*}
    W_{\text{theo}} &= \SI{0.66}{\electronvolt} \\
    W_{1,\%} &= \SI{9.6(33)}{\percent}  \\
    W_{2,\%} &= \SI{10.2(27)}{\percent}
\end{align*}
Die Abweichungen liegen also auch dicht beieinander, was Sinn macht, da mit ansteigender Heizrate keine Veränderung der Aktivierungsenergien einhergehen sollte.
Unsicherheiten bei der Berechnung ergeben sich aus den gewählten Intervallen, die die Untergrundrate von den eigentlich gemessenen Strömen abgrenzen soll.
Diese Intervalle sind schwierig zu definieren, da keine klaren Grenzen erkennbar sind.
Bei genauerer Betrachtung der einzelnen Energien wird gefunden, dass die Methode über die Polarisation genauer ist.
\\
\newline
Die errechnete Relaxationszeit weicht stark von dem Theoriewert ab.  
Die Diskrepanz, nicht nur durch die verschiedenen Heizraten, sondern auch durch die unterschiedlichen Methoden, sorgen vorallem 
in der Berechnung der Relaxationszeit für einen großen Unterschied, da sie mit einer Exponentialfunktion gewichtet werden.
\begin{align*}
    \tau_{0,\text{theo}} &= \SI{4e-14}{\second} \\
    \tau_{1,\%} &= \SI{96(9)}{\percent}  \\
    \tau_{2,\%} &= \SI{70(50)}{\percent}
\end{align*}
Auffällig sind zudem die unterschiedlich hohen Peaks des gemessenen Stroms in Abhängigkeit der Temperatur. 
Diese folgen daraus, dass bei erhöhter Heizrate die meisten Dipole noch nicht relaxiert sind und die Temperatur weiter stetig steigt.
Also liegt eine hohe Anzahl an immer noch orientierten Dipolen vor, die alle wegen der hohen Temperatur fast zeitgelich relaxieren und sich 
als hohen Strom messen lassen.
\\
\newline
Grundsätzlich lässt sich sagen, dass sich mit den uns vorgestellten Methoden gut die Aktivierungsenergien errechnen lassen können.
Gleiches gilt weniger für die Relaxationszeit bei der hohe Abweichungen zur Theorie auftreten. 

























%mit einer Steigung die annährend $\SI{1}{\degree\per\minute}$ und $\SI{2}{\degree\per\minute}$ 
