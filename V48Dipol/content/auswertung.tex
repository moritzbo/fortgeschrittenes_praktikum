\section{Auswertung}

In dem Versuch wurde die Dipolrelaxation für zwei unterschiedliche Heizraten $b$ untersucht. Da die Temperatur zum Teil manuell nachgeregelt werden musste, bietet es sich an die Heizrate über
die gemessenen Temperatur-Zeit-Abstände zu bestimmen. In der Abbildung \ref{fig:a1} sind die Temperatur-Zeit Verläufe dargestellt. Durch eine Lineare Regression lässt sich die Heizrate $b$ jeweils bestimmen zu
\begin{align*}
    b_1 &= \SI{1.009(9)}{\kelvin\per\minute},\\
    b_2 &= \SI{2.050(23)}{\kelvin\per\minute}.\\
\end{align*}
\begin{figure}
    \centering
    \includegraphics[width=\textwidth]{build/temperatur.pdf}
    \caption{Gemessene Temperaturverläufe beider Heizraten $b_1$ und $b_2$.
            }
    \label{fig:a1}
\end{figure}
\subsection{Temperatur-Strom-Kurve}

Aus den gemessenen Daten \ref{tab:WhoIsBigInJapan?1} und \ref{tab:WhoIsBigInJapan?2} lassen sich nun die Temperatur-Strom Kurven auftragen. Dies ist in den Abbildungen \ref{fig:a2} und \ref{fig:a3} für die verschiedenen Heizraten gezeigt. Zusätzlich wird der Untergrund durch einen Exponentialfit angedeutet. 
Dieser besitzt also die Form
\begin{equation*}
I(\text{T}) = m \cdot \text{exp} \left(nT\right).
\end{equation*}
Aus den Strom-Temperatur Wertepaaren ergeben sich somit die Parameter $m$ und $n$ für die jeweilige Heizrate angedeutet durch die Indizes zu
\begin{align*}
m_1 &= \SI{4.050(2121)e-9}{\pico\ampere}\\  
n_1 &= \SI{0.059(18)}{\per\kelvin}\\ 
m_2 &= \SI{1.309(581)e-7}{\pico\ampere}\\ 
n_2 &= \SI{0.057(15)}{\per\kelvin}\\ 
\end{align*}

\begin{figure}
    \centering
    \includegraphics[width=\textwidth]{build/strom1.pdf}
    \caption{Gemessene Strom-Temperaturverläufe bei einer Heizrate \newline $ b_1 = \SI{1.009(9)}{\kelvin\per\minute}$ mit Exponentialfit der Untergrundrate.
            }
    \label{fig:a2}
\end{figure}
\begin{figure}
    \centering
    \includegraphics[width=\textwidth]{build/strom2.pdf}
    \caption{Gemessene Strom-Temperaturverläufe bei einer Heizrate \newline $b_2 = \SI{2.050(23)}{\kelvin\per\minute}$ mit Exponentialfit der Untergrundrate.
            }
    \label{fig:a3}
\end{figure}
Ebenfalls lassen sich noch die zuvor gefundenen Ausgleichsfunktionen für die Untergrundrate von den Werten abziehen. Dies ist in den Abbildung \ref{fig:b1} und \ref{fig:b2}
dargestellt.
\begin{figure}
    \centering
    \includegraphics[width=\textwidth]{build/untegrrund1.pdf}
    \caption{Gemessene Strom-Temperaturverläufe bei einer Heizrate \newline $ b_1 = \SI{1.009(9)}{\kelvin\per\minute}$ mit abgezogener Untergrundrate.
            }
    \label{fig:b1}
\end{figure}
\begin{figure}
    \centering
    \includegraphics[width=\textwidth]{build/untegrrund2.pdf}
    \caption{Gemessene Strom-Temperaturverläufe bei einer Heizrate \newline $b_2 = \SI{2.050(23)}{\kelvin\per\minute}$ mit abgezogener Untergrundrate.
            }
    \label{fig:b2}
\end{figure}
\subsection{Berechnung der Aktivierungsenergie über den Strom} 

Die Aktivierungsenergie $W$ lässt sich wie in dem Abschnitt \ref{sec:strom} beschrieben nach Gleichung \eqref{eqn:theo6} berechnen. Dabei werden die Strommessdaten logarithmiert und gegen 
die reziproke Temperatur aufgetragen. Wichtig hierbei ist, dass nur ein kleiner Abschnitt von Temperaturen verwendet werden darf, da
die Näherung \eqref{eqn:theo5} noch gelten soll. Dafür wurde hier die Linke Flanke, der in Abbildung \ref{fig:b1} und \ref{fig:b2} dargestellten ersten Maxima verwendet. Eine Ausgleichsgerade der Form
\begin{equation}
\text{ln}(I(T)) = a \cdot \left( \frac{1}{T}\right) + c,
\end{equation}
liefert die Fitparameter für die einzelnen Heizraten
\begin{align*}
a_1 &= \SI{-7374.38(52868)}{\kelvin},\\
c_1&= \SI{29.02(214)}{},\\
a_2 &= \SI{-9171.08(24711)}{\kelvin},\\
c_2&= \SI{36.33(100)}{}.
\end{align*}
Aus Gleichung \eqref{eqn:theo6} lässt sich nun der Zusammenhang zur Aktivierungsenergie $W$ durch folgende Beziehung erreichen
\begin{equation}
W_i = - k_B \cdot a_i.
\end{equation}
Somit folgt für die Aktivierungsenergien 
\begin{align}
W_1 &= \SI{0.64(5)}{\electronvolt},\\
W_2 &= \SI{0.79(2)}{\electronvolt}.
\end{align}
In den Abbildungen \ref{fig:a4} und \ref{fig:a5} sind die logarithmierten Messdaten, sowie die Ausgleichsgeraden für die jeweiligen Heizraten dargestellt.
\begin{figure}
    \centering
    \includegraphics[width=\textwidth]{build/benergie1.pdf}
    \caption{Logarithmierter Stromverlauf bei einer Heizrate \newline $b_1 = \SI{1.009(9)}{\kelvin\per\minute}$ der linken Flanke des ersten Maximums.
            }
    \label{fig:a4}
\end{figure}
\begin{figure}
    \centering
    \includegraphics[width=\textwidth]{build/benergie2.pdf}
    \caption{Logarithmierter Stromverlauf bei einer Heizrate \newline $b_2 = \SI{2.050(23)}{\kelvin\per\minute}$ der linken Flanke des ersten Maximums.
            }
    \label{fig:a5}
\end{figure}

\subsection{Berechnung der Aktivierungsenergie über die Polarisation}
Die zweite Methode die Aktivierungsenergien zu berechnen, verwendet die Polarisationsmethode nach Abschnitt \ref{sec:pola}. Hierfür muss eine numerische 
Integration des Integrals in Gleichung \eqref{eqn:idkwhat} durchgeführt werden. Eine Möglichkeit stellt eine Integration durch
Aufsummieren von Trapeznäherungen. Hierbei gilt 
\begin{equation}
I_{\int_{T(x_1)}^{T(x_n)}} = \sum_{i=1}^{n-1} \left( (x_{i+1} - x_{i}) \frac{f(x_{i+1}) + f(x_{i})}{2}\right).
\end{equation}
Wird dies nun für passende Temperaturintervalle für die unterschiedlichen Heizraten durchgeführt können die Integrale so genähert werden.
Bei der Wahl der Intervalle muss stets die genannte Bedingung aus \ref{sec:pola} erfüllt sein. Es bieten sich die Intervalle
\begin{align*}
\increment T_1 &= \{\SI{240.85}{\kelvin}\text{, ...,} \underbrace{\SI{268.95}{\kelvin}}_{T*_1}\}, \\
\increment T_2 &= \{\SI{239.15}{\kelvin}\text{, ...,} \underbrace{\SI{280.35}{\kelvin}}_{T*_2}\}, \\
\end{align*}
an. Die Integrale bilden für die Zwischenschritte der Temperatur, in den vorher genannten Intervallen, Wertepaare zusammen mit der reziproken Temperatur nach 
Gleichung \eqref{eqn:idkwhat}. Diese wurden in den Abbildungen \ref{fig:lol1} und \ref{fig:lol2} aufgetragen. Die Austrittsarbeit ergibt sich dabei wieder durch eine lineare 
Regression.
\begin{figure}
    \centering
    \includegraphics[width=\textwidth]{build/lol1.pdf}
    \caption{Integrierte Messwerte der ersten Heizrate \newline $b_1 = \SI{1.009(9)}{\kelvin\per\minute}$ gegen die reziproke Temperatur.
            }
    \label{fig:lol1}
\end{figure}
\begin{figure}
    \centering
    \includegraphics[width=\textwidth]{build/lol2.pdf}
    \caption{Integrierte Messwerte der ersten Heizrate \newline $b_2 = \SI{2.050(23)}{\kelvin\per\minute}$ gegen die reziproke Temperatur.
            }
    \label{fig:lol2}
\end{figure}
Die Ausgleichsfunktion der Form
\begin{equation}
\text{ln} \left( \frac{\int_{\text{T}}^{\infty} I(\text{T'}) \text{dT'} }{I(\text{T})b}\right) = a \cdot \frac{1}{T} + c
\end{equation}
liefert die Parameter
\begin{align*}
    a_1 &= \SI{9551.42(25316)}{\kelvin},\\
    c_1&= \SI{-35.32(96)}{},\\
    a_2 &= \SI{7902.61(41423)}{\kelvin},\\
    c_2&= \SI{-28.54(160)}{}.
\end{align*}
Woraus sich nun die Aktivierungsenergien aus dem Zusammenhang zu Gleichung \eqref{eqn:idkwhat} bestimmen lassen als
\begin{align}
    W_1 &= \SI{0.82(2)}{\electronvolt},\\
    W_2 &= \SI{0.68(4)}{\electronvolt}.
\end{align}
 
\subsection{Bestimmung der charakteristischen Relaxationszeit}
Für die Bestimmung der charakteristischen Relaxationszeit wird die Gleichung \eqref{eqn:hier} verwendet. Dazu werden aus den Abbildungen \ref{fig:b1} und \ref{fig:b2} die 
Temperaturwerte des ersten Maximums abgelesen zu
\begin{align}
    T_{1\text{,max}} &= \SI{255.45}{\kelvin},\\
    T_{2\text{,max}}&= \SI{260.35}{\kelvin}.
\end{align}
Nun wurden verschiedene Aktivierungsenergie $W$ für die jeweiligen Heizraten berechnet und somit liefern beide einen anderen Wert für $\tau_0$.
Eingesetzt in Gleichung \eqref{eqn:hier} ergibt sich für die erste Heizrate $ b_1 = \SI{1.009(9)}{\kelvin\per\minute}$:
\begin{align}
    \tau_{0\text{,}1\text{,}\alpha} &= \SI{2.058(4834)e-12}{\second},\\
    \tau_{0\text{,}1\text{,}\beta} &= \SI{4.513(4211)e-16}{\second}.
\end{align}
Sowie für die zweite Heizrate $b_2 = \SI{2.050(23)}{\kelvin\per\minute}$: 
\begin{align}
    \tau_{0\text{,}2\text{,}\alpha} &= \SI{1.839(1686)e-15}{\second},\\
    \tau_{0\text{,}2\text{,}\beta} &= \SI{2.877(5299)e-13}{\second}.
\end{align}
Dabei steht der zweite Index für die Heizrate und der dritte Index für die Methode. Die Stromdichtenmethode wird mit $\alpha$ und die Polarisationsmethode mit $\beta$
gekennzeichnet.

\subsection{Darstellung des Temperaturverlaufs der Relaxationszeit}
Nun wurden alle wichtigen Parameter für Gleichung \eqref{eqn:lol1} für jeweils zwei Heizraten bestimmt.
Da es für $\tau_0$ und die Aktivierungsenergie $W$ jeweils zwei Werte gibt, werden diese gemittelt
\begin{align}
    \overline{\tau_{0\text{,}1}} &= \SI{1.029(2417)e-12}{\second},\\
    \overline{\tau_{0\text{,}2}} &= \SI{1.448(2650)e-13}{\second}, \\
    \overline{W_{1}} &= \SI{0.7301(269)}{\electronvolt}, \\
    \overline{W_{2}} &= \SI{0.7351(224)}{\electronvolt}. 
\end{align}
Nach Einsetzen in Gleichung \eqref{eqn:lol1} können die resultierenden Funktionen in Abbildung \ref{fig:whatthefuck} dargestellt werden.
\begin{figure}
    \centering
    \includegraphics[width=\textwidth]{build/lolol1.pdf}
    \caption{Temperaturabhängigkeit der Relaxationszeit bei verschiedenen Heizraten.
            }
    \label{fig:whatthefuck}
\end{figure}
