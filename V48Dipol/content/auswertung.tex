\section{Auswertung}

In dem Versuch wurde die Dipolrelaxation für zwei unterschiedliche Heizraten $b$ untersucht. Da die Temperatur zum Teil manuell nachgeregelt werden musste, bietet es sich an die Heizrate über
die gemessenen Temperatur-Zeit-Abstände zu bestimmen. In der Abbildung \ref{fig:a1} sind die Temperatur-Zeit Verläufe dargestellt. Durch eine Lineare Regression lässt sich die Heizrate $b$ jeweils bestimmen zu
\begin{align*}
    b_1 &= \SI{1.009(9)}{\kelvin\per\second},\\
    b_2 &= \SI{2.050(23)}{\kelvin\per\second}.\\
\end{align*}
% \begin{figure}
%     \centering
%     \includegraphics[width=0.8\textwidth]{build/?.png}
%     \caption{Gemessene Temperaturverläufe beider Heizraten $b_1$ und $b_2$.
%             }
%     \label{fig:?}
% \end{figure}
\subsection{Temperatur-Strom-Kurve}

Aus den gemessenen Daten ... lassen sich nun die Temperatur-Strom Kurven auftragen. Dies ist in Abbildung \ref{fig:a2} gezeigt. Zusätzlich wird der Untergrund durch einen Exponentialfit angedeutet. 
Dieser besitzt also die Form
\begin{equation*}
I(\text{T}) = m \text{exp} \left(nT\right).
\end{equation*}
Aus den Strom-Temperatur Wertepaaren ergeben sich somit die Parameter $m$ und $n$ für die jeweilige Heizrate angedeutet durch die Indizes zu
% \begin{align*}
%     m_1 &= \SI{}{\pico\ampere}\\  
%     n_1 &= \SI{}{\per\kelvin}\\ 
%     m_2 &= \SI{}{\pico\ampere}\\ 
%     n_2 &= \SI{}{\per\kelvin}\\ 
% \end{align*}

% \begin{figure}
%     \centering
%     \includegraphics[width=0.8\textwidth]{build/?.png}
%     \caption{Gemessene Strom-Temperaturverläufe beider Heizraten $b_1$ und $b_2$ mit Exponentialfit der Untergrundrate.
%             }
%     \label{fig:?}
% \end{figure}

\subsection{Berechnung der Aktivierungsenergie}

Die Aktivierungsenergie $W$ lässt sich wie in dem Abschnitt ...