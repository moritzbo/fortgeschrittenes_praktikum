\section{Auswertung}

In dem Versuch wurde die Dipolrelaxation für zwei unterschiedliche Heizraten $b$ untersucht. Da die Temperatur zum Teil manuell nachgeregelt werden musste, bietet es sich an die Heizrate über
die gemessenen Temperatur-Zeit-Abstände zu bestimmen. In der Abbildung \ref{fig:a1} sind die Temperatur-Zeit Verläufe dargestellt. Durch eine Lineare Regression lässt sich die Heizrate $b$ jeweils bestimmen zu
\begin{align*}
    b_1 &= \SI{1.009(9)}{\kelvin\per\minute},\\
    b_2 &= \SI{2.050(23)}{\kelvin\per\minute}.\\
\end{align*}
\begin{figure}
    \centering
    \includegraphics[width=\textwidth]{build/temperatur.pdf}
    \caption{Gemessene Temperaturverläufe beider Heizraten $b_1$ und $b_2$.
            }
    \label{fig:a1}
\end{figure}
\subsection{Temperatur-Strom-Kurve}

Aus den gemessenen Daten ... lassen sich nun die Temperatur-Strom Kurven auftragen. Dies ist in den Abbildungen \ref{fig:a2} und \ref{fog:a3} für die verschiedenen Heizraten gezeigt. Zusätzlich wird der Untergrund durch einen Exponentialfit angedeutet. 
Dieser besitzt also die Form
\begin{equation*}
I(\text{T}) = m \cdot \text{exp} \left(nT\right).
\end{equation*}
Aus den Strom-Temperatur Wertepaaren ergeben sich somit die Parameter $m$ und $n$ für die jeweilige Heizrate angedeutet durch die Indizes zu
\begin{align*}
m_1 &= \SI{4.050(2121)e-9}{\pico\ampere}\\  
n_1 &= \SI{0.059(18)}{\per\kelvin}\\ 
m_2 &= \SI{1.309(581)e-7}{\pico\ampere}\\ 
n_2 &= \SI{0.057(15)}{\per\kelvin}\\ 
\end{align*}

\begin{figure}
    \centering
    \includegraphics[width=\textwidth]{build/strom1.pdf}
    \caption{Gemessene Strom-Temperaturverläufe bei einer Heizrate \newline $ b_1 = \SI{1.009(9)}{\kelvin\per\minute}$ mit Exponentialfit der Untergrundrate.
            }
    \label{fig:a2}
\end{figure}
\begin{figure}
    \centering
    \includegraphics[width=\textwidth]{build/strom2.pdf}
    \caption{Gemessene Strom-Temperaturverläufe bei einer Heizrate \newline $b_2 = \SI{2.050(23)}{\kelvin\per\minute}$ mit Exponentialfit der Untergrundrate.
            }
    \label{fig:a3}
\end{figure}


\subsection{Berechnung der Aktivierungsenergie über den Strom} 

Die Aktivierungsenergie $W$ lässt sich wie in dem Abschnitt \ref{sec:strom} beschrieben nach Gleichung \eqref{eqn:theo6} berechnen. Dabei werden die Strommessdaten logarithmiert und gegen 
die reziproke Temperatur aufgetragen. Wichtig hierbei ist, dass nur ein kleiner Abschnitt von Temperaturen verwendet werden darf, da
die Näherung \eqref{eqn:theo5} noch gelten soll. Dafür wurde hier die Linke Flanke, der in Abbildung ... dargestellten ersten Maxima verwendet. Eine Ausgleichsgerade der Form
\begin{equation}
\text{ln}(I(T)) = a \cdot \left( \frac{1}{T}\right) + c,
\end{equation}
liefert die Fitparameter für die einzelnen Heizraten
\begin{align*}
a_1 &= \SI{-7374.38(52868)}{\kelvin},\\
c_1&= \SI{29.02(214)}{},\\
a_2 &= \SI{-9171.08(24711)}{\kelvin},\\
c_2&= \SI{36.33(100)}{}.
\end{align*}
Aus Gleichung \eqref{eqn:theo6} lässt sich nun der Zusammenhang zur Aktivierungsenergie $W$ durch folgende Beziehung erreichen
\begin{equation}
W_i = - k_B \cdot a_i.
\end{equation}
Somit folgt für die Aktivierungsenergien 
\begin{align}
W_1 &= \SI{0.64(5)}{\electronvolt},\\
W_2 &= \SI{0.79(2)}{\electronvolt}.
\end{align}
In den Abbildungen \ref{fig:a4} und \ref{fig:a5} sind die logarithmierten Messdaten, sowie die Ausgleichsgeraden für die jeweiligen Heizraten dargestellt.
\begin{figure}
    \centering
    \includegraphics[width=\textwidth]{build/benergie1.pdf}
    \caption{Logarithmierter Stromverlauf bei einer Heizrate \newline $b_1 = \SI{1.009(9)}{\kelvin\per\minute}$ der linken Flanke des ersten Maximums.
            }
    \label{fig:a4}
\end{figure}
\begin{figure}
    \centering
    \includegraphics[width=\textwidth]{build/benergie2.pdf}
    \caption{Logarithmierter Stromverlauf bei einer Heizrate \newline $b_2 = \SI{2.050(23)}{\kelvin\per\minute}$ der linken Flanke des ersten Maximums.
            }
    \label{fig:a5}
\end{figure}

\subsection{Berechnung der Aktivierungsenergie über die Polarisation}
Die zweite Methode die Aktivierungsenergien zu berechnen, verwendet die Polarisationsmethode nach Abschnitt \ref{sec:pola}. Hierfür muss eine numerische 
Integration des Integrals in Gleichung \eqref{eqn:idkwhat} durchgeführt werden. Eine Möglichkeit stellt eine Integration durch
Aufsummieren von Trapeznäherungen. Hierbei gilt 