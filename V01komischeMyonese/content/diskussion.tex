\section{Diskussion}

Bei der Justierung der Verzögerungszeit vor der Koinzidenzschaltung zeigt sich ein Plateau in einem Intervall, welches von der eingestellten Impulsbreite
abhängt. Dabei ist das Maximum an Zählungen pro 10 Sekunden bereits bei nur kleinen Verzögerungen aufzufinden. Es zeigt sich, je größer die eingestellte Impulsbreite,
desto größer auch die Halbwertsbreite. Dieses Ergebnis war zu erwarten. 
\\
\newline
Der Zusammenhang zwischen angezeigtem Kanal und Zeitdifferenz lies sich sehr gut bereits durch einige Werte angeben. Dies liegt an dem quasi perfekten linearen Verlauf, welcher
in Abbildung \ref{fig:333} zu erkennen ist.
\\
\newline
Bei der statistischen Berechnung der Untergrundrate stellt sich ein nur kleiner Effekt, im Verhältnis zur allgemeinen Größenordnung der Zählraten, heraus. Dies liegt vor allem an der nur sehr geringen Anzahl
an Startsignalmessungen pro Suchzeit $T$. 
\\
\newline
Die Berechnung der Lebensdauer der Myonen, über die Methode der kleinsten Quadrate, liefert ein im Vergleich zum Theoriewert sehr gutes Ergebnis. Der Theoriewert
\cite{myontime} beträgt
\begin{equation}
\tau_{\text{theo}} = \SI{2.197}{\micro\second}.
\end{equation}
Somit liegt die prozentuale Abweichung zum Theoriewert bei 
\begin{equation}
\tau_{\si{\percent}} = \SI{3.29(87)}{\percent}.
\end{equation}