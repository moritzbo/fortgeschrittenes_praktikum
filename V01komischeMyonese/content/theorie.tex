\section{Theorie}
\subsection{Leptonen und Elementarteilchen}
Zu den Elementarteilchen gehören die Quarks und Leptonen. Diese sind, anders als die Hadronen \enquote{elementar}, dass heißt nicht weiter aus anderen Teilchen zusammengesetz. 
Quarks sind sechsteilig in drei Generationen gespalten und bilden durch verschiedene Kombinationen die kurzlebigen Hadronen. Frei kommen diese im Gegensatz zu den Leptonen nicht vor. 
Diese Leptonen sind also gut, durch zum Beispiell Zerfallprozesse, und entsprechendem Aufwand detektierbar. Charakteristisch für diese Teilchen ist zum einen die \enquote{Schwache -} als auch die 
\enquote{Elektromagnetische Wechselwirkung}. Zusätzlich wirkt, wie auf alles was Masse besitzt, die Gravitation. Diese ist in ihrer Größenordnung den andren Kräften stark unterlgen und wird im folgenden
nicht beachtet. Analog zu den Quarks finden sich auch die Leptonen in drei Generationen zusammen. 
\begin{table}
    \centering
    \caption{In der Tabelle sind die Leptonen in ihren verschiedenen Generation aufgetragen. In den Spalten sind von links nach rechts sind die Teilchen und entsprechenden Antiteilchen zu sehen.
            Das zu jeder Generation passende Neutrino ist rechts neben dem Namensgebenen Lepton zu finden. Die Nomenklatur bietet nur bei Elektronen an das Antiteilchen mit eigenem Namen, dem \enquote{Positron}, zu beschreiben. In den 
            anderen Generationen ist es üblich den postiv geladenen Partner mit dem präfix \enquote{anti - } zu nennen.}
    \label{tab:1}
    \begin{tabular}{c | c c c c | c c c}
        \toprule
        Generation &\multicolumn{4}{c}{Teilchen} & \multicolumn{3}{c}{Antiteilchen}  \\
        \midrule
        1      &       Elektron & $\text{e}^\text{-}$   &     Elektron-Neutrino  &  $\nu_\text{e}$  &    (Positron) & $\text{e}^\text{+}$  &   $\nu_\text{e}^\text{+}$  \\
        2      &      Myon & $\mu^{\--}$         &           Myon-Neutrino       &  $\nu_\mu$       &            & $\mu^\text{+}$       &   $\nu_\mu^\text{+}$        \\
        3      &      Tauon & $\tau^{\--} $         &       Tau-Neutrino         &  $\nu_\tau$      &            & $\tau^\text{+}$      &   $\nu_\tau^\text{+}$       \\
    \end{tabular}
\end{table}
Das einzig stabile Lepton ist das Elektron und das Positron. Ein Elektron mit der Masse $m_\text{e}$ ist um den Faktor 206 leichter als ein Myon und 3491 mal leichter als ein Tauon. 
Sowohl die Myonen als auch die Tauonen, mit Antiteilchen, haben eine statistische Lebensdauer. Maßgebliche Eigenschaften der
Leptonen ist das Verhalten nach dem Pauli-Prinzip. Dieses schreibt einen Assamble von gleichen Teilchen vor, welcher Zustand für jedes Individuum möglich ist. Außerdem sind Leptonen 
ununterscheidbar.  Das Pauli-prinzip und die Ununterscheidbarkeit führen zu Fermi-Dirac Statistik, die Aussagen über Fermionen, Teilchen mit dem Spin $1/2$, trifft. Alle drei Bedingungen dieser Statistik
treffen auf Leptonen zu, diese seinen also auch den Fermionen zuzuordnen.  
\\
\newline
Experimentell wurde bei Reaktionen mit Leptonen ein Defizit von Energie beobachtet was zur richtigen Annahme führte, es gibt zusätzlich zum eigentlich Tielchen noch ein weiters. 
Dieses weitere Teilchen, die Neutrinos, wurden später auch nachgewiesen und ihnen kann eine Masse, die kleiner als $m_e$ ist, zugewiesen werden.
Quantitativ lassen sich die einzelnen Leptonen, vorallem bei Zerfällen, durch die Leptonenzahl beschreiben. Diese soll eine Erhaltungsgröße sein
die folgich vor und nach dem Zerfall gleich ist. So sei die elektronische Leptonenzahl $\text{l}_\text{e}$ einer Generation genau dann $1$, wenn es sich um ein Elektron handelt. 
Analog hat das Myon eine myonische Leptonenzahl $\text{l}_\mu$ von $1$ aber eine elektronische von $0$. Bei Ladungswechseln zum Antiteilchen ändert sich das Vorzeiche der Quantenzahl.
Demnach sieht der Zerfall von Myonen und Anitmyonen wie folgt aus:
\begin{equation*}
    \mu^\text{-} \rightarrow \text{e}^\text{-} + \bar{\nu_\text{e}} + \nu_\mu.
\end{equation*}
Analog für den Antiteilchen Zerfall:
\begin{equation*}
    \mu^\text{+} \rightarrow \text{e}^\text{+} + \nu_\text{e} + \bar{\nu_\mu}.
\end{equation*}
Die Leptonenzahl bleibt auf beiden Seiten des Zerfalls erhalten. 

\subsection{Entstehung von Myonen durch kosmischer Strahlung}
Die Erde ist konstant der kosmischen Strahlung ausgesetzt. Diese entsteht nicht nur durch die erdnahe Sonne, sondern auch durch eine Vielzahl anderer kosmischer Phänomene die weit jenseits unseres Sonnensystem ihren Ursprung haben, nicht zuletzt der
durch den Urknall erzeugten Hintergrundstrahlung. Ein großteil der hochenergetischen Strahlung besteht jedoch aus Protonen die stark mit der Atmosphäre, abhängig von der Dichte, wechselwirken.
Es kommt, dass diese Protonen in der Atmosphäre unter anderem zu Pionen, oder $\pi$-Mesonen, zerfallen und diese wiederem zu Myonen. 
