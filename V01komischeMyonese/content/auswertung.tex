\section{Auswertung}

In der folgenden Versuchsauswertung werden für die Methode der kleinsten Quadrate als auch lineare Regressionen, die Bibliotheken NumPy \cite{numpy} und SciPy \cite{scipy} 
in Python verwendet. Die Fehlerrechnungen wurden ebenfalls über die Bibliothek \enquote{uncertainties} \cite{uncertainties} bestimmt.

\subsection{Justierung der Verzögerung}
Da an dem Szintillator zwei Photomultiplier angeschlossen sind, müssen diese Signale zeitlich aneinander angepasst an die Koinzidenzschaltung gelangen. An beiden 
Leitungen lässt sich durch eine variable Kabellänge eine Verzögerung generieren, sodass dies gewährleistet ist. Im folgenden wurden die ankommenden Spannungsimpulse
über einen Diskriminator angepasst. Dazu wurde eine hinreichend große Schwellspannung zum herausfiltern des Rauschens und eine Impulsbreite eingestellt. Die Verzögerungsanpassung
wurde im Folgenden für zwei unterschiedliche Impulsbreiten von $\SI{10}{\nano\second}$ und $\SI{20}{\nano\second}$ durchgeführt. 
\\
Die Messreihen mit unterschiedlich eingebauten Verzögerungen und den dazu gemessenen Zählraten pro $\SI{10}{\second}$ hinter der Koinzidenz, sind in den Tabellen
\ref{tab:MessreiheDelay20ns} und \ref{tab:MessreiheDelay10ns} notiert.
\\
Zusätzlich gilt eine Poissonverteilung der Messwerte, da es sich um Zählraten handelt. Die Fehler der einzelnen Werte werden also mit $\sqrt{N}$ angegeben.
Von Interesse ist nun die Halbwertsbreite der Verzögerungseinstellung. Diese lässt sich über zwei Lineare Regressionen an den jeweiligen Flanken für je eine Impulsbreite bestimmen.
\\
Die Lineare Ausgleichgerade hat dabei die Form
\begin{equation*}
N(t) = at + b.
\end{equation*}
In den Tabelle \ref{tab:112233} und \ref{tab:332211} sind die ermittelten Werte der Parameter aufgelistet.

\begin{table}
    \centering
    \caption{Parameter der Linearen Regression für die Bestimmung der Halbwertsbreite bei einer Impulsbreite von $\SI{10}{\second}$.} 
    \label{tab:112233}
    \begin{tabular}{c | c c }
        \toprule
        Flanke & a [$\si{\per\nano\second}$] & b \\
        \midrule
            Links    &      $\SI{15.87(69)}{}$     &   $\SI{302.49(843)}{}$      \\               
            Rechts    &     $\SI{-16.33(85)}{}$     &   $\SI{306.60(1121)}{}$    \\ 
    \end{tabular}
\end{table}

\begin{table}
    \centering
    \caption{Parameter der Linearen Regression für die Bestimmung der Halbwertsbreite bei einer Impulsbreite von $\SI{10}{\second}$.} 
    \label{tab:332211}
    \begin{tabular}{c | c c }
        \toprule
        Flanke & a [$\si{\per\nano\second}] $ & b \\
        \midrule
            Links    &      $\SI{12.85(61)}{}$      &    $\SI{132.21(406)}{}$  \\               
            Rechts    &     $\SI{-15.26(88)}{}$      &  $\SI{150.28(605)}{}$   \\ 
    \end{tabular}
\end{table}

Aus den Messreihen \ref{tab:MessreiheDelay20ns} und \ref{tab:MessreiheDelay10ns} lassen sich nun die Maxima ablesen und in die jeweiligen halben Maxima in die Ausgleichgeraden einsetzten.
Wenn diese nach $t$ umgeformt werden, ergeben sich die Werte
\begin{align*}
t_{\text{links, }10} &= \SI{-5.07(40)}{\nano\second},\\
t_{\text{rechts, }10} &= \SI{5.46(51)}{\nano\second},\\
t_{\text{links, }20} &= \SI{-11.06(72)}{\nano\second},\\
t_{\text{rechts, }20} &= \SI{11.00(90)}{\nano\second}.\\
\end{align*}
Aus den Differenzen der Beträge folgen somit die beiden Halbwertsbreiten
\begin{align*}
T_{10} &= \SI{10.53(64)}{\nano\second},\\
T_{20} &= \SI{22.06(115)}{\nano\second}.\\
\end{align*}
In den Abbildungen \ref{fig:111} und \ref{fig:222} sind die Messwerte mit poissonverteiltem Fehlerbalken, sowie Ausgleichgeraden und Halbwertsbreite für beide Impulsbreiten 
aufgetragen.
\begin{figure}
    \centering
    \includegraphics[width=\textwidth]{build/plot11.pdf}
    \caption{Gemessene Impulse pro 10 Sekunden mit Fehlerbalken, Ausgleichsgeraden und Halbwertsbreite für eine Impulsbreite von $\SI{20}{\nano\second}$.} 
    \label{fig:111}
\end{figure}
\begin{figure}
    \centering
    \includegraphics[width=\textwidth]{build/plot12.pdf}
    \caption{Gemessene Impulse pro 10 Sekunden mit Fehlerbalken, Ausgleichsgeraden und Halbwertsbreite für eine Impulsbreite von $\SI{10}{\nano\second}$.} 
    \label{fig:222}
\end{figure}

\subsection{Kalibrierung des Vielkanalanalysators}
Um die Spannungsimpulse hinter dem TAC einer Zeit zuzuordnen, werden mit einem Doppelimpulsgenerator feste Zeitabstände generiert. Die am Rechner abgelesenen Kanäle pro Zeitabstand $\increment t$
sind in der Tabelle \ref{tab:1122332211} eingetragen. Sie zeigen einen linearen Verlauf. Es lässt sich also durch die Messwerte eine lineare Ausgleichgerade bestimmen. Durch diese kann dann einem
beliebigen Kanal eine Zeitdifferenz $\increment t$ zugeorndet werden.
\\
Die Form der Ausgleichgeraden sieht folgendermaßen aus
\begin{equation}
\increment t (\text{C}) = a  \text{C} + b.
\end{equation}
Dabei steht $\text{C}$ für den jeweiligen Channel. 
Es ergeben sich die folgenden Parameter
\begin{align*}
    a &= \SI{2.750(1)}{\nano\second\per{\text{Channel}}},\\
    b &= \SI{0.286(226)}{\micro\second}.\\
\end{align*}
Die gemessenen Werte aus der Tabelle \ref{tab:1122332211} sind zusammen mit der Ausgleichgerade in der Abbildungen \ref{fig:333} aufgetragen.
\begin{figure}
    \centering
    \includegraphics[width=\textwidth]{build/plot2.pdf}
    \caption{Zuordnung von Zeitintervallen zu den Kanälen über eine Lineare Regression der Messwerte.} 
    \label{fig:333}
\end{figure}