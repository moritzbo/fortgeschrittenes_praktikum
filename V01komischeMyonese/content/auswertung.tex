\section{Auswertung}

In der folgenden Versuchsauswertung werden für die Methode der kleinsten Quadrate als auch lineare Regressionen, die Bibliotheken NumPy \cite{NumPy} und SciPy \cite{SciPy} 
in Python verwendet. Die Fehlerrechnungen wurden ebenfalls über die Bibliothek \enquote{uncertainties} \cite{uncertainties} bestimmt.

\subsection{Justierung der Verzögerung}
Da an dem Szintillator zwei Photomultiplier angeschlossen sind, müssen diese Signale zeitlich aneinander angepasst an die Koinzidenzschaltung gelangen. An beiden 
Leitungen lässt sich durch eine variable Kabellänge eine Verzögerung generieren, sodass dies gewährleistet ist. Im folgenden wurden die ankommenden Spannungsimpulse
über einen Diskriminator angepasst. Dazu wurde eine hinreichend große Schwellspannung zum herausfiltern des Rauschens und eine Impulsbreite eingestellt. Die Verzögerungsanpassung
wurde im Folgenden für zwei unterschiedliche Impulsbreiten von $\SI{10}{\nano\second}$ und $\SI{20}{\nano\second}$ durchgeführt. 
\\
Die Messreihen mit unterschiedlich eingebauten Verzögerungen und den dazu gemessenen Zählraten pro $\SI{10}{\second}$ hinter der Koinzidenz, sind in den Tabellen
\ref{tab:MessreiheDelay20ns} und \ref{tab:MessreiheDelay10ns} notiert.
\\
Zusätzlich gilt eine Poissonverteilung der Messwerte, da es sich um Zählraten handelt. Die Fehler der einzelnen Werte werden also mit $\sqrt{N}$ angegeben.
Von Interesse ist nun die Halbwertsbreite der Verzögerungseinstellung. Diese lässt sich über zwei Lineare Regressionen an den jeweiligen Flanken für je eine Impulsbreite bestimmen.
\\
Die Lineare Ausgleichgerade hat dabei die Form
\begin{equation*}
N(t) = at + b.
\end{equation*}
In den Tabelle \ref{tab:112233} und \ref{tab:332211} sind die ermittelten Werte der Parameter aufgelistet.

\begin{table}
    \centering
    \caption{Parameter der Linearen Regression für die Bestimmung der Halbwertsbreite bei einer Impulsbreite von $\SI{10}{\second}$.} 
    \label{tab:112233}
    \begin{tabular}{c | c c }
        \toprule
        Flanke & a [$\si{\per\nano\second}$] & b \\
        \midrule
            Links    &      $\SI{15.87(69)}{}$     &   $\SI{302.49(843)}{}$      \\               
            Rechts    &     $\SI{-16.33(85)}{}$     &   $\SI{306.60(1121)}{}$    \\ 
    \end{tabular}
\end{table}

\begin{table}
    \centering
    \caption{Parameter der Linearen Regression für die Bestimmung der Halbwertsbreite bei einer Impulsbreite von $\SI{10}{\second}$.} 
    \label{tab:332211}
    \begin{tabular}{c | c c }
        \toprule
        Flanke & a [$\si{\per\nano\second}] $ & b \\
        \midrule
            Links    &      $\SI{12.85(61)}{}$      &    $\SI{132.21(406)}{}$  \\               
            Rechts    &     $\SI{-15.26(88)}{}$      &  $\SI{150.28(605)}{}$   \\ 
    \end{tabular}
\end{table}

Aus den Messreihen \ref{tab:MessreiheDelay20ns} und \ref{tab:MessreiheDelay10ns} lassen sich nun die Maxima ablesen und in die jeweiligen halben Maxima in die Ausgleichgeraden einsetzten.
Wenn diese nach $t$ umgeformt werden, ergeben sich die Werte
\begin{align*}
t_{\text{links, }10} &= \SI{-5.07(40)}{\nano\second},\\
t_{\text{rechts, }10} &= \SI{5.46(51)}{\nano\second},\\
t_{\text{links, }20} &= \SI{-11.06(72)}{\nano\second},\\
t_{\text{rechts, }20} &= \SI{11.00(90)}{\nano\second}.\\
\end{align*}
Aus den Differenzen der Beträge folgen somit die beiden Halbwertsbreiten
\begin{align*}
T_{10} &= \SI{10.53(64)}{\nano\second},\\
T_{20} &= \SI{22.06(115)}{\nano\second}.\\
\end{align*}