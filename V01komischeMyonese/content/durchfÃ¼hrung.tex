\section{Durchführung}
Der in \ref{fig:aufbauSCHEMATISCH} gezeigte Aufbau wird schrittweise aufgebaut, was ermölglicht, die einzelnen Elemente
frühzeitig zu kalbirieren. Zu Beginn werden die Diskriminatoren eingestellt um ein Rauschen zu vermeiden und die Impulsbreite zu definieren.
Hierbei wird mit einem Schraubenzieher eine kleine Stellschraube in dem Bauelement verstellt. Die Schwelle für einkommende Signale
wird durch die \enquote{Threshhold}-Schraube geändert und die Breite durch \enquote{width}. Das Ziel ist eine möglichst flache, ungestörte Linie 
neben den einfallenden Impulsen am Oszillograhpen zu beobachten. Dierser Impuls soll eine Breite von $\SI{10}{\micro\second}$ und $\SI{20}{\nano\second}$ haben.
\\
\newline
Es folgen die \enquote{Delays}, deren Verzögerungen angepasst werden müssen. Als Impuls werden einfallende Myonen und deren Zerfälle 
benutzt, die in ihrerem Signal keinen unterscheidbaren Unterschied haben. Weiter wird der Aufbau so verändert, dass die Koinzidenzschaltung direkt in eine Zählereinheit läuft. 
Zu erwarten sind bei einer perfekten Abstimmung der Delays maximale Zählraten. Es gilt analog, dass eine schlechte Abstimmung zu niedrigen Raten führt. 
Die Verzögerungen an den Delays sind in nano Sekunden als zweier Potenzen, wobei zusätzlich $0.5$ Möglich ist, einstellbar und lassen sich addieren. So kann pro Delay
ein Maximum von $\SI{63.5}{\nano\second}$ Verzögerungen erreicht werden. Zur bestimmung des optimalen verhätnis werden beide Delays auf einen gleichen Wert über $\SI{0}{\nano\second}$ eingestellt und von dort 
in mehreren Schritten variert. Der Zähler soll für eine Länge von $10$ Sekunden suchen und das Ergebnis gilt es zu notieren.
Nach hinreichender Beobachtung werden die Delays für eine maximale Zählraten eingestellt.
\\
\newline
Die restlichen Elemente werden hinzugenommen und an dem \enquote{Monoflop} wird die Suchzeit $\increment T_s = \SI{15}{\micro\second}$ eingestellt. 
Um die Auswertung durchen den TAC zu ermöglichen muss vor der eigentlichen Messreihe der Doppelimpulsgernator alleine an die Koinzidenzschaltung
geschlossen werden. Der Aufbau ab dem Ausgang der Koinzidenzschaltung ändert sich dabei nicht. 
Jetzt werden mehrere Doppelimpuslse bei bekannter zeitlichen Länge erzeugt und durch den TAC ausgewertet. Die entsprechenden Kanäle können am PC, der mit der TAC-Einheit verbunden ist, 
ausgelesen und notiert werden. So ist bekannt, welchem Kanal eine Impulsdauer zugeordnet ist. 
Final wird der Sizillator wieder an die Koinzidenzschaltung angeschlossen und die eigentlich Messreihe kann beginnen. 
Es müssen alle Messvorgänge zur gleichen Zeit gestartet werden. Dazu gehört die Zählung der einfallenden Teilchen und die Messung der Lebensdauer am PC.
Nach einer beliebigen Zeit, wahlweise groß um die Mittellung zu verbessern, wird der Messvorgang gestoppt und die Daten entnommen.