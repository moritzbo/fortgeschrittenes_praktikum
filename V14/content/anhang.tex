\newpage
\section{Anhang}
\begin{table}
    \centering
    \caption{Messwerte der gezählten Einfälle (i.e. Counts $N$ pro $\SI{300}{\second}$) bei Durchleuchtung verschiedener Würfel aus unterschiedlichen Materialien aber konstantem
            Kanal 157 am Ausleseprogramm.} 
    \label{tab:1}
    \begin{tabular}{c c c c c c}
    \toprule
    Index & KEIN Würfel  & Ummantelung  & $\text{Würfel}_1 $ &  $\text{Würfel}_2 $  & $\text{Würfel}_3$ \\
    \midrule
1    &  4765   &   4582  &  3865 &   158  &   1236  \\
2    &         &       4510  &      3819 &       129  &   1325  \\
3    &         &       4460  &      3815 &       244  &   1205  \\
4    &         &            &          &           &   949   \\   
5    &         &            &          &           &   769   \\
6    &         &            &          &           &   1049  \\
7    &         &            &          &           &   3625  \\
8    &         &            &          &           &   168   \\
9    &         &            &          &           &   3596  \\
10   &         &            &          &           &   991   \\
11   &         &            &          &           &   776   \\
12   &         &            &          &           &   916   \\
\end{tabular}
\end{table}



\begin{table}
    \centering
    \caption{Theoriewerte für eine Auswahl an Materialen aus \cite{...} für etwaige Vergleiche zu Messwerten.
            Die Annahme, ein Würfel möge aus Holz bestehen für zu der letzten zeile, welche eben diese Daten angibt. Es wird davon ausgegangen \cite{holz}, dass
            Holz durch ein Kombination aus 50\% Kohlenstoff, 44\% Sauerstoff und 6\% Wasserstoff gebildet ist. } 
    \label{tab:1}
    \begin{tabular}{c c c c c c c}
    \toprule
     & $\rho [\si{\gram\per\centi\meter^3}]$ & $\sigma_{\text{compton}} [\si{\centi\meter^2\per\gram}] $ & $\sigma_{\text{photo}} [\si{\centi\meter^2\per\gram}] $  & $\sigma_{\text{gesamt}} [\si{\centi\meter^2\per\gram}]$ &  $\mu_{\text{gesamt}}[\si{\per\centi\meter}]$ \\
    \midrule
    ~ &Alluminium  &2,700 & 0,074   & 0,000 & 0,074 &  0,201 \\        
    ~ &Blei        &11,340& 0,060   & 0,044 & 0,104 &  1,174 \\    
    ~ &Eisen       &7,860 & 0,072   & 0,001 & 0,073 &  0,571 \\    
    ~ &Messing     &8,600 & 0,070   & 0,001 & 0,071 &  0,610 \\    
    ~ &Delrin      &1,410 & 0,082   & 0,000 & 0,082 &  0,117 \\    
    (Annahme) & Holz &  0,47 & 0,111  & 0,000 & 0,111 & 0,052 \\
\end{tabular}
\end{table}



