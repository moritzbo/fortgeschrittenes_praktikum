\section{Auswertung}
Kern der Auswertung ist die Bestimmung der Absorptionskoeffizienten und dammit die Deutung der einzelnen Elementarwürfel aus Würfel 4.
Dafür wird in den ersten Schritten Aufschluss über die ersten drei Würfel durch Mittellung der eindeutigen Absorptionskoeffizienten erlangt. 
In passender Literatur \cite{...} finden sich Referenzwerte dieser Koeffizienten wodurch sich die hier verwendeten Materialien finden lassen.  
Folglich wird die Auswertung in vier Teile aufgebaut, die sich jeweils mit einem Würfel befaasst.

\subsection{Würfel 1}
Die Messreihen für den ersten Würfel dienen, da dieser nur aus Ummandlung besteht, der Bestimmung des Absorptionskoeffizienten für Alluminium.
Die Strecke die der Strahl durch das Material bewältigen muss ist die doppelte Dicke der Ummandlung also
\begin{equation}
    \increment x = \SI{2}{\milli\meter}.
\end{equation}
Aus dem Gesetz von D'bruderwie schreib ich dich \eqref{...} folgt nach Umformen zu 
\begin{equation}
\mu = - \text{ln} \left( \frac{I(\increment x)}{I_0} \right) \frac{1}{\increment x}
\end{equation}