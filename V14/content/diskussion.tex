\section{Diskussion}

Bei den ersten drei Messreihen, also von Würfel 1 bis 3 zeigen sich genaue Ergebnisse mit niedrigen Abweichungen von 
\begin{align*}
\increment_{\text{Alu}} &= \SI{32.78}{\percent}, \\
\increment_{\text{Pb}} &= \SI{8.02}{\percent}, \\
\increment_{\text{Holz}} &= \SI{5.30}{\percent}.
\end{align*}
Es stellt sich heraus, dass der erste Würfel die erwartete Aluminiumummantelung besitzt und ein Holz, sowie ein Bleiwürfel vorliegt.
Die hohe Genauigkeit liegt vorallem daran, dass die Absorptionskoeffizienten hier alle durch lineare Strahlverläufe durch den Würfel bestimmt werden konnten. Außerdem entsteht das Ergebnis durch eine Mittelung über drei verschiedene Werte welche 
dies ebenfalls genauer machen. In dem Fall des vierten Würfels, der aus neun unbekannten Elementarwürfeln zusammengebaut war, ergibt sich eine größere Abweichung, durch die insgesamt sechs Diagonalmessungen durch das Objekt. Diese liefern
deutlich größere Abweichung, was zum Einen an der Schwierigkeit des Ausrichtens liegt und an der eventuell leicht veränderten Intensität $I_0$ schräg durch den Würfel.
\\
\newline
Trotzdem lassen sich die neun Elementarwürfeln relativ gut zu den vorher bestimmten Würfeln zuordnen. Die Abweichungen schwanken dabei allerdings sehr stark. Eine Mögliche Verbesserung der statistischen Abweichung ist beispielsweise durch 
eine Erhöhung der Messungen möglich. Die wohl größte Unsicherheit liegt aber an der Würfelausrichtung und der Breite der Strahlenquelle im Strahlengang. Hier lässt sich eventuell eine schmalere Blende einsetzen, unter der allerdings die Intensität
vermindert wird.