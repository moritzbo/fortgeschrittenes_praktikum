\section{Diskussion}

Die Messung des magnetischen Feldes innerhalb des Elektromagneten ergab ein breites Maximum am Ort der Probenstelle. Dieses lag bei $\SI{480}{\milli\tesla}$ und blieb über den Versuchsverlauf relativ konstant.
Auf Grund des großen Plateus lässt sich das auf die Probe wirkende Magnetfeld als homogen und konstant annehmen. Kleine Abweichungen können durch das ständige Umpolen des Elektromagneten bei der Messung der Faraday-Rotation entstehen.
\\
\newline
Bei der Messung der Winkel für die unterschiedlichen Proben und Filter sind teils große Diskrepanzen zur linearen Ausgleichsgerade zu erkennen. Die Messgenauigkeit hängt dabei auch stark von der Ablesegenauigkeit am Oszilloskop im Zusammenspiel
mit Differenz- und Selektivverstärker ab. Das Minimum am Oszilloskop konnte somit immer unterschiedlich gut abgelesen werden.
Außerdem ist der Messaufbau sehr sensibel für jegliche Erschütterungen, da das Licht immer gleich auf die Sensoren einfallen sollte.
Etwaige Änderungen in der Raumhelligkeit sorgen ebenfalls für Abweichungen.
Zusätzlich sollte stets darauf geachtet werden, dass der Selektivverstärker nicht überladen wird, denn in diesem Zeitraum können keine brauchbaren Messwerte entnommen werden.
\\
\newline
Die gefundenen effektiven Massen $m^*$ weichen um den Literaturwert $m_{\text{lit}}^{*} = 0,06m_e$ \cite{GaAs}, abhängig von der Dotierung folgendermaßen ab
\begin{align}
m^{\%}_{1{,}2} &= \SI{70.79(6217)}{\percent},\\
m^{\%}_{2{,}3} &= \SI{131.40(6140)}{\percent}.\\
\end{align}
Zu erkennen sind also relativ große Abweichungen, welche größtenteils systematischer Natur sind. Es gibt insgesamt, was bereits an Formel \eqref{eqn:wichtig}
erkennbar ist, viele Fehlerquellen die hier mit eingehen. Für die Größenordnung der Messgröße sind diese allerdings annehmbar.