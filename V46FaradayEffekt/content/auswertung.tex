\newpage
\section{Auswertung}
Die Auswertung sei in zwei Abschnitte unterteilt, die einmal auf die Messung des Magnetfelds eingeht und zum 
anderen die Bestimmung der effektiven Masse nach \ref{theo} versucht. 
\subsection{Bestimmung des magnetischen Feld}
Die Messdaten der Kraftflussdichte vom angelegten, magnetischen Feld sind in Abbildung \ref{fig:mag}
bei entsprechender Entfernung zur Probeleerstelle aufgetragen.
\begin{figure}
    \centering
    \includegraphics[width=\textwidth]{build/plot1.pdf}
    \caption{In der Abbildung sind die Messwerte aus der Bestimmung des magnetischen Feldes aufegtragen. 
            Deutlich zu sehen ist ein Platoo um die Probenleerstelle herum. Es lässt sich also annehmen, dass
            um die leige ein annährend konstantes Feld an.}
    \label{fig:mag}
\end{figure}
\\
Die Messdaten verpsrechen, dass die Probe einer Feldstärke von 
\begin{equation}
    \SI{408}{\tesla}
\end{equation}
ausgesetzt waren.
\begin{figure}
    \centering
    \includegraphics[width=0.7\textwidth]{build/plot2.pdf}
    \caption{hallo}
\end{figure}

\begin{figure}
    \centering
    \includegraphics[width=0.7\textwidth]{build/plot3.pdf}
    \caption{hallo}
\end{figure}

\begin{figure}
    \centering
    \includegraphics[width=0.7\textwidth]{build/plot4.pdf}
    \caption{hallo}
\end{figure}