\newpage
\section{Auswertung}
Die Auswertung sei in zwei Abschnitte unterteilt, die einmal auf die Messung des Magnetfelds eingeht und zum 
anderen die Bestimmung der effektiven Masse nach \ref{theo} versucht. 
\subsection{Bestimmung des magnetischen Feld}
Die Messdaten der Kraftflussdichte vom angelegten, magnetischen Feld sind in Abbildung \ref{fig:mag}
bei entsprechender Entfernung zur Probeleerstelle aufgetragen.
\begin{figure}
    \centering
    \includegraphics[width=\textwidth]{build/plot1.pdf}
    \caption{In der Abbildung sind die Messwerte aus der Bestimmung des magnetischen Feldes aufegtragen. 
            Deutlich zu sehen ist ein Platoo um die Probenleerstelle herum. Es lässt sich also annehmen, dass
            um die leige ein annährend konstantes Feld an.}
    \label{fig:mag}
\end{figure}
\\
Die Messdaten verpsrechen, dass die Probe einer Feldstärke von 
\begin{equation}
    \SI{408}{\tesla}
\end{equation}
ausgesetzt waren.
\begin{figure}
    \centering
    \includegraphics[width=0.7\textwidth]{build/plot2.pdf}
    \caption{hallo}
    \label{maf2}
\end{figure}

\begin{figure}
    \centering
    \includegraphics[width=0.7\textwidth]{build/plot3.pdf}
    \caption{hallo}
\end{figure}

\begin{figure}
    \centering
    \includegraphics[width=0.7\textwidth]{build/plot4.pdf}
    \caption{hallo}
\end{figure}


\subsection{Bestimmung der effektiven Masse}
Um die effektive Masse zu bestimmen werden die einzelnen, gemessenen Winkel unter denen ein Minimum gefunden wurde
normiert, also als Verhältnis von Winkel zu Länge der Probe dargestellt.
Für die drei Proben sind die Messwerte in der Abbildung \ref{fig:maf2} gezeigt, wobei der Winkel 
jeweils gegen das Quadrat der, durch den Filter, erlaubten Wellenlänge aufgetragen ist.
Die gleichen Filter wurden bei jeder Probe verwendet, weswegen es sich nicht anbietet die Werte durch eine \enquote{Fit} zu modellieren.
Sie können also direkt, um die Differenz zu der undotierten Probe zu bekommen, voneinander Abgezogen werden.
In der Tabelle \ref{tabelle} sind die Differenzen aufgelistet.