\section{Aufbau}

\begin{figure}
    \centering
    \includegraphics[width=0.9\textwidth]{bilder/Aufbau.png}
    \caption{Schematischer Aufbau einer Apperatur um die Drehung der Polarisation durch den Faraday-Effekt zu messen \cite{skript}.}
    \label{fig:aufbau}
\end{figure}

Der Aufbau beginnt mit einer Halogen-Lampe als Lichtquelle, die also das darauf folgende Element bestrahlt. Das 
emittierte Spektrum einer solchen Lampe befindet sich größtenteils im infraroten Bereich, was den Halbleiter und ihrer 
Resonanzfrequenz gut entgegen kommt. Mit einem Kodensor, also einer Sammellinse, wird das Licht eingefangen und so gebrochen, 
dass das Licht nun parallel zur optischen Achse der Sammellinse weiterläuft. 
Die Lichtquelle wird genau im Brennpunkt des Kondensors platziert, um dessen Funktion zu garantieren.
\\
\newline
In einem darauffolgenden \enquote{Lichtzerhacker} wird durch Modulation weitesgehend vermieden, dass weitere leuchtende Quellen im Raum
auf das Empfängersignal einwirken. Die bis hierhin modifizierten Lichstrahlen werden anschließend in einem
\enquote{Glan-Thompson-Prisma} aus Kalkspat gebrochen. \\
Ein solches, doppelbrechendes Prisma ist in der Lage unpolarisiertes Licht zu polarisieren. Dadurch entsehen 
aus einem Lichtstrahl, zwei weitere mit senkrechtzueinander stehender Polarisation. Einzig der erzeugte Strahl, 
der parallel zur optischen Achse läuft wird verwendet. Unterstützt wird der Verlauf dieser Achse durch eine metallische Schiene,
auf der die einzlnen Elemente montiert werden können.
\\
\newline
Ein Konstantstromgerät, mit der Möglichkeit die Polung zu ändern, ist an einem Elektromagneten angeschlossen um ein konstantes (durch Gleichstrom) Magnetfeld
zu erzeugen. Dieses Magnetfeld soll später eine Probe umschließen und möglichst gleichmäßig darum verteilt sein. Damit die Verteilung 
auch bei Eingriffen in den Aufbau möglichst konstant bleibt, bietet es sich an, den Magneten mit auf die Schiene zu bauen. Eine Einbuchtung 
darin lässt die einzelnen Proben einführen. Diese sind dort senkrecht zum Lichtstrahl fesgehalten, der wiederum ungehindert zwischen den Hälften des Magneten 
strahlen kann. Die Probe, in der schließlich der Farady-Effekt stattfindet, weist eine relativ hohe Dichte auf, weswegen viel Intensität des Lichtes verloren geht. 
So ist die Wahl vom nötigen Interferenzfilter weitesgehend limitiert auf eine Stelle, nachdem die Strahlen durch die Probe gelangt sind. 
Gewählt wurde im Aufbau eine Platzierung des Filters direkt hinter dem Magneten. Das Licht erfährt ab hier keine weiteren Intensitätsverluste und kann hier problemlos
auf die gewollte Wellenlänge angepasst werden.
\\
\newline
Um das nun linear polarisierte -und zudem gedrehte Licht auswerten zu können, Bedarf es noch ein weiteres Doppelprisma, welches wieder auf der optischen Achse, also der Schiene,
montiert wird. Hier wird das Licht entsprechend der Ausrichtung der Polarisation entweder auf beide, durch Reflektion und Transmission,
möglichen Strahlenwege verteilt, oder im extremen Fall nur in eine Richtung weitergegeben. Die lineare Polarisation bleibt natürlich weiterhin erhalten.
\\
\newline
Analog zur Lichtquelle am Anfang, die im Brennpunkt des Kondensor platziert ist, werden nun die beiden möglichen Strahlen
über eine andere Sammellinse eingefangen. Dieses mal befinden sich zwei Photowiderstände hinter den Linsen im Brennpunkt.
Die Photowiderstände sind varibale Widerstände, die mit zunehmender Intensität durch bestrahlendes Licht abnehmen.
Ein Kondensator nach jeweils einem Photowiderstand schwächt mit der Modulation durch den Lichtzerhacker die auftretenden Störeffekte 
der Widerstände ab. Die beiden Kondensatoren laufen am andere Ende zusammen in ein Element namens \enquote{Differenzverstärker},
welcher mit ansteigender Spannungsdifferenz der beiden Kondensatoren selbst auch eine höhere Spannung ausgibt. Seine Spannung ist also 
quasi proportional zur einlaufenden Differenz. Mit einem Selektivverstärker als Zwischenelement, ist final ein Oszillograph angeschlossen
der für den Experimentator die gemessene Spannung anschaulich dastellen kann. Durch den Differenzverstärker ist also ein Minimum eben dann zu erkennen, wenn das erste 
Doppelprisma genau um $\SI{90}{\degree}$ zum zweiten verschoben ist.



\section{Durchführung}
Die Durchfühung unterteilt sich in mehrere Schritte, welche unter der Justage, also die Ausrichtung der Elemente
und der eigentlichen Messung unterscheiden. Nach der optimalen Konfiguration der Elemente 
gilt es die eigentliche Messreihe vorzunehmen, also die Bestimmung der Winkel. Abschließend 
bietet es sich an das magnetische Feld mit einer Hall-Sonde abzufahren und so zu messen.

\subsection{Justage der Apparatur}
Der Aufbau wie in \ref{fig:aufbau} gezeigt soll hier hinreichend geprüft und eventuell korrigiert werden.
Dabei wird damit begonnen das System ohne Probe und Filter zu untersuchen. So kann optisch festgestellt werden, 
ob die Prismen das Licht weiter in die richtige Richtung, also zum einen an das andere Doppelprisma selbst und zum anderen die Photowiderstände, weiterleiten.
Dazu werden die Blenden an den Photowiderständen entfernt und die Winkel zum Prsima eben so eingestellt, das
ein deutlich einfallendes und zentriertes Licht auf die Widerstände fällt.
Anschließend folgt das Finden einer passenden Resonatorfrequenz für den Selektivverstärker,
welche im weiteren Verlauf des Versuches beibehalten wird. Sodann sollten die einzelnen Photowiderstände zusammen 
an den Differenzverstärker geschlossen werden, dieser wiederum an den Selektivverstäker,
welcher die Spannung liefert, die am Oszillographen ausgelesen wird. 
Um sicherzustellen, dass der Aufbau für etwaige Messungen eingestellt ist, wird das drehbare Prisma am Anfang des 
Aufbaus solange gedreht, bis am Oszillographen ein klares Miniumum erkennbar ist. 
Die Symmetrie der Prismen verspricht die gleiche Intensität beider Lichtstrahlen, die vom zweiten 
Prisma ausgehen, wenn die zwei Prismen auf der optischen Achse um $\SI{90}{\degree}$
zueinander gedreht sind. Sollte also ein zweites, deutliches Minimum nach weiteren $\SI{90}{\degree}$ zum ersten Minimmum erkennbar sein, %notiz 
kann der Aufbau als annehmbar angenommen werden.

\subsection{Messreihen zur bestimmung der Drehwinkel}
Mit Einsetzten der Probe und Filter wir auch das magnetische Feld mit einer Spannung von
$\SI{20}{\volt}$ und der Lichtzerhacker angeschaltet. Die eigentliche Messreihe gestaltet sich dann so,
dass bei angelegtem magnetischem Feld und eingeführter Probe und Filter der Winkel des ersten Prismas
für ein auftretendes Minimum am Oszillographen gesucht wird. 
Anschließend wir die Spannung für das magnetische Feld auf $\SI{0}{\volt}$ gesetzt
und die Stromversorgung umgepolt. Mit wieder voll anlegender Spannung an der Versorgung wird das Minimum erneut gesucht.
Der Vorgang wiederholt sich für alle Proben mit jedem möglichen Interferenzfilter.
Die Probe selbst und die Filter nehmen dem Lichtstrahl viel Intensität, wesewegen es bei etwaigen 
Wechseln sich empfiehlt mit einem Gegenstand den Strahl vor den Widerständen zu unterbrechen.
So wird vermieden, dass die erzeugten Spannungen zu groß werden und die Elemente überlasten.

\subsection{Messreihe zur Bestimmung der Magnetfeldstärke}
Zuletzt wird gemessen, wie stark die Proben dem Magnetfeld ausgesetzt waren. 
Dazu werden die hier unnötigen Photowiderstände samt zweitem Prisma entfernt und mit einer Hall-Sonde 
getauscht. Diese ist fest auf der Schiene zu montieren wobei sie trotzdem, dank eines verschiebbaren 
Aufsatz bewegt werden kann. Zudem ist an diesem Aufsatz eine Skala angebracht mit der man also der Magnetfeldstärke
eine Entfernung zuordnen kann.

