\section{Theorie}

\subsection{Mikro - und Elektromagnetische Wellen }
Mikrowellen beschreiben elektromagnetische Wellen, die sich in einem vorher definierten Frequenzspektrum von 1 bis 300 \si{\giga\hertz} befinden. 
Sie sind im Elktromagnetischem Spektrum eher dem langwelligen Bereich zuzuordnen, weit jenseits des für uns sichtbaren Lichts.
Elektromagnetische Wellen lassen sich hinreichend durch die Maxwell Glecihungen beschreiben, ohne jedoch dabei die Teilchen Eigenschaften von Photonen zu berücksichtigen. Im Folgenene Versuch wird sich außschließlich auf die Welleneigenschaften bzeogen.
Charakteristisch für all diese Wellen ist die im Vakuum ungehinderte Geschwindigkeit sich fortzubewegen, besser bekannt als Lichtgeschwindigkeit. 
Mit dieser Geschwindigkeit \si{\c} #????
, oder im Falle eines Mediums der entsprechend reduzierten, lässt sihc nun die Wellenlänge mithilfe der Frequenz \si{\nu} bestimmen.
\begin{equation}
\lambda = c/{\nu}
\end{equation}
Das Magnetische Feld verhält sich analog zum elktrischen, sthet aber zu jedem Zeitpunkt senkrecht auf seinem elktrischen Partner. 