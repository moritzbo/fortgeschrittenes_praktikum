\section{Aufbau und Versuchsdurchführung}

Der grundlegende Aufbau bleibt in den einzelnen Versuchsabschnitten gleich, dabei werden teilweise lediglich Geräte hinzugefügt oder getauscht. 
Für die Messung der verschiedenen Moden in Abhängigkeit von der Reflektorspannung $U_{R}$ wird der allgemeinste Aufbau nach Abbildung ... verwendet.

Zu sehen ist zunächst das Reflexklystron welches an ein Netzteil angeschlossen ist, die durch das Klystron erzeugten Mikrowellen werden über einen Hohlleiter aus dem Resonator ausgekoppelt und durch einen Einweggleichrichter geführt. 
Dieser dient dazu, dass keine Mikrowellen in das Klystron zurückreflektiert werden und somit für Ungenauigkeiten sorgen. 
Dahinter werden sowohl Frequenzmessgerät, als auch eine Dämpfungsquelle geschaltet. Die Messung geschieht über einen Detektor welcher mit einem digitalen Oszilloskop verbunden wird. 
Über den gesamten Versuchszeitraum wird die Kathodenspannung konstant bei $U_{K} = \SI{6.3}{\volt}$ belassen.

\subsection{Messung der einzelnen Moden}

Um die Moden der durch den Klystron erzeugten Mikrowellen zu messen wird eine konstante Dämpfung von $\SI{30}{\decibel}$ am Dämpfungsglied eingestellt. 
Dabei lässt sich die angebrachte Dämpfungskurve verwenden. Ein Channel des Oszilloskop wird nun
mit der am Netzteil einstellbaren Reflektorspannung $U_R$, über ein BNC-Kabel am \enquote{$\SI{0}{}$-$\SI{30}{\volt}$, $\SI{50}{\hertz}$ $\sim$}
