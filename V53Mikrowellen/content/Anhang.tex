\section{Anhang}

\begin{table}
    \centering
    \caption{Gemessene Refelektorspannung mit zugehörigen Amplitdenmaxima der Detektorspannung und Frequenz beim Abtasten nach einzelnen Moden in einem Spannungintervall von $50$-$250 \si{\volt}$.} 
    \label{tab:Messreihe11}
    \begin{tabular}{c | c c c}
        \toprule
        Mode & Reflektorspannung $U_{R} [\si{\volt}] $ & Detektorspannung $ U_{Mikrowelle} [\si{\milli\volt}] $ & Frequenz f $[\si{\mega\hertz}]$ \\
        \midrule
        1      &      212         &       150              &        9036 \\
        2      &      132         &       170              &        9038 \\
        3      &      80          &       140              &        9045 \\
    \end{tabular}
\end{table}
 
\begin{table}
    \centering
    \caption{Gemessene Reflektorspannung und Frequenz an den linken und rechten Rändern der einzelnen Moden in einem Spannungintervall von $50$-$250 \si{\volt}$.}
    \label{tab:Messreihe12}
    \begin{tabular}{c | c c | c c}
        \midrule
        ~ &  \multicolumn{2}{c}{linker Rand} & \multicolumn{2}{c}{rechter Rand} \\
        \toprule
        Mode & Spannung $U_{lM} [\si{\volt}] $ & Frequenz $f_{lM}[\si{\mega\hertz}]$ & Spannung $U_{rM} [\si{\volt}] $ & Frequenz $f_{rM}[\si{\mega\hertz}]$ \\
        \midrule
        1     &     220         &            9053            &                200          &              9015      \\
        2      &    152         &            9075            &                115          &              9017      \\
        3       &   82          &            9087            &                60           &              8987      \\
    \end{tabular}
\end{table}

       \begin{table}
           \centering
           \caption{Abstände der gefunden Minima mit der errechneten Differenz.}
           \label{tab:Messreihe21}
           \begin{tabular}{c c | c}
               \toprule
               Minimum & Abstand $d$ $[\si{\milli\meter}]$ & Differenz der Abstände $ \Delta d$ $ [\si{\milli\meter}]$ \\
               \midrule
               1 & 51,00 & \multirow{2}{*}{24,5} \\
               2 & 75,50 &  \\
           \end{tabular}
       \end{table}


\begin{table}
    \centering
    \caption{Messreihe zur Bestimmung und Überprüfung der Dämüfung des Dämpfgliedes mit Hilfe der SWR.}
    \label{tab:Messreihe22}
    \begin{tabular}{c c c}
        \toprule
       SWR - Ausschlag $[\si{\dB}]$ & Mikrometereinstellung $[\si{\milli\meter}]$ & Dämpfung aus Eichkurve $[\si{\dB}]$ \\
        \midrule
        0   & ~ & ~ \\
        2   & ~ & ~ \\
        4   & ~ & ~ \\
        6   & ~ & ~ \\
        8   & ~ & ~ \\
        10  & ~ & ~ \\        
    \end{tabular}
\end{table}


\begin{table}
    \centering
    \caption{Stehwellenmessung direkt mit dem SWR - Messggerät. }
    \label{tab:Messreihe31}
    \begin{tabular}{c c c c c}
        \toprule
        ~ &  3 & 5 & 7 & 9 \\
        \midrule
       SWR & 1,06 & 1,3 & 2,2 & 2,8 \\   
     \end{tabular}
\end{table}

\begin{table}
    \centering
    \caption{Stehwellenmessung mit der 3-dB-Methode.}
    \label{tab:Messreihe32}
    \begin{tabular}{c c c c c c }
        \toprule
        Abstand $d_1$ $[\si{\milli\meter}]$ & Abstand $d_2$ $[\si{\milli\meter}]$ & $\text{Min}_1 [\si{\milli\meter}]$&  $\text{Min}_2 [\si{\milli\meter}]$ & $\lambda_g [\si{\milli\meter}]$ & SWR \\
        \midrule
        &  &  &  & & \\   
     \end{tabular}
\end{table}

\begin{table}
    \centering
    \caption{Stehwellenmessung mit der Abschwäche-Methode.}
    \label{tab:Messreihe32}
    \begin{tabular}{c c c c}
        \toprule
        $A_1 [\si{\milli\meter}]$ & $A_2 [\si{\milli\meter}]$ & $A_2 - A_1 [\si{\milli\meter}]$ & SWR \\
        \midrule
        &  &  & \\   
     \end{tabular}
\end{table}
