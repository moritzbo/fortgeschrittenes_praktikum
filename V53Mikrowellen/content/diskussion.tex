\section{Diskussion}

Zunächst lässt sich folgendes an dem Versuchsaufbau und der Durchführung anmerken. An jeder Stelle bei der ein weiteres Messgerät
eingebaut ist entstehen mögliche Reflektionsstellen, die mögliche Ungenauigkeiten liefern. Daher ist es 
unteranderem wichtig die Verstärkung am SWR-meter in Zusammenhang mit der einstellbaren Dämpfung am 
Dämpfungsglied richtig zu kalibrieren. 
\\
Die einzelnen Bauelemente sollten ebenfalls immer gut verschraubt sein. Dies ist leicht zu bemerken wenn der Hohlleiter bewegt wird und 
anschließend am SWR-meter eine große Schwankung entsteht. Unterstützt werden kann dies mit Halterungen unter den Bauelementen, um ein Absinken und dadurch auftretende
Spalte zu vermeiden.
\\
Über die Genauigkeit von beispielsweise Einweggleichrichter, sowie Abschluss lässt sich nicht viel sagen. Die Werte bei
herausgedrehter Dämpfung scheinen allerdings nicht stark gestört zu sein.
\\
\newline
Bei der Bestimmung der einzelnen Moden ließen sich die Mikrowellenspannungen gut am Oszilloskop ablesen und deuten auf 
keine großen systematischen Fehler. Das gleiche gilt für die Frequenz nach dem Frequenzmessgerät, welches gut eingestellt werden konnte.
Die Ergebnisse zeigen wie zu erwarten war, höhere Modenzahlen für kleinere Reflektorspannungen. Die Bandbreite nimmt mit steigender Mode zu, wie in 
der Tabelle \ref{tab:111} zu erkennen ist. Eine Modenanalyse mit diesem Messprinzip scheint also sinnvolle Ergebnisse zu liefern. Dennoch sind drei Messwerte
pro Mode das mindeste für die Bestimmung eines brauchbaren Ausgleichpolynoms und könnte somit beliebig erhöht werden.
\\
\newline
Die Frequenz und Wellenlängenmessungen beruhen alle auf nur zwei Messwerten, sorgen somit also für besonders hohe Ungenauigkeiten. Es gibt keinen Grund diesen
Wert als nicht tragbar anzusehen, allerdings ist es sinnvoll, beliebig viele weitere Minima zu messen und über diese zu mitteln. Die errechnete Hohlleiterfrequenz
von $f_h = \SI{8.968(10)}{\giga\hertz}.$ scheint allerdings sehr plausibel. Die prozentuale Abweichung zur Messung mit dem Frequenzmessgerät beträgt $\increment f = \SI{0.75(11)}{\percent}$.
\\
\newline
Bei der Auswertung der Dämpfungskurve zeigt sich deutlich, dass das SWR-meter bei herausgedrehter Dämpfung nicht perfekt auf $\SI{0}{\decibel}$ kalibriert wurde. 
Dies liegt daran, dass die Dämpfungskurve exponentiell ansteigt und somit die ersten Werte nahezu Null anzeigen. Der Nullpunkt lässt sich somit nur sehr
ungenau bestimmen, was in der Grafik \ref{123} zu sehen ist. Eine Möglichkeit um den Offset zu verringern wäre beispielsweise eine feste Dämpfung am
Dämpfungsglied einzustellen und diese am SWR-meter so umzustellen, dass sie als genau diese angezeigt wird. Danach kann zurück zu $\SI{0}{\decibel}$ gedreht werden und
es lässt sich ein nur geringer Offset vermuten. 
In unserem Fall beträgt der Offset knapp $\SI{8}{\decibel}$. Diese Werten liegen dennoch sehr gut auf der ermittelten Ausgleichskurve.
\\
\newline
Die drei verwendeten Methoden zur Bestimmung des Stehwellenverhältnisses zeigen unterschiedliche Genauigkeiten in verschiedenen Messbereichen. Dabei ist die direkte Methode besser verwendbar
für kleinere Dämpfungen und die 3 dB, sowie Abschwächermethode für größere geeignet. Dies ist unteranderem an dem Messwert für die Gleitschraubentiefe von $\SI{9}{\milli\meter}$ 
bei der direkten Methode im Vergleich zu den anderen erkennbar. Sie liefert dort ein SWR von $\infty$ während die beiden anderen Methoden noch konkrete Werte geben. 
Über die Notwendigkeit der Messwerte für SWR > 10 lässt sich allerdings streiten, da elektrische Leiter mit solchen SWR-Werten in der Anwendung bereits unbrauchbar sind.
\begin{table}
    \centering
    \caption{Vergleich der Stehwellenverhältnisse bei einer Gleitschraubentiefe von $ \SI{9}{\milli\meter}$.}
    \label{tab:1}
    \begin{tabular}{c  c || c }
        \toprule
        Methode & $\text{SWR}$ & Abweichung \hspace{0.5cm} $\text{SWR}_{\text{3dB}}$ $\sim$ $\text{SWR}_{\text{A}}$ \hspace{0.5cm} [$\%$]  \\
        \midrule
         3 dB-Methode        & 8.987 & \multirow{2}{*}{36.38} \\
         Abschwächer-Methode & 14.125 & \\ 
         Direkt              & $ \infty$ & \\ 
     \end{tabular}
\end{table}
\begin{flushleft}
Die prozentuale Abweichung auf das SWR zwischen 3 dB, und Abschwächermethode beträgt $\SI{36.38}{\percent}$. 
\end{flushleft}